\documentclass[a4paper,11pt]{article}
\usepackage[utf8]{inputenc}
\usepackage{hyperref}

%opening

\title{}
\author{}

\begin{document}

\maketitle

\section{Abstract}

\section{Introduction}
This experiment represents a replication of a part of an experiment done by Yoon et al. (2016) \cite{yoon2016talking}. This study was concerned with politeness, a technique we use in everyday life conversations with different intentions. One example could be that your friend is terrible at cooking, but in order not to offend him you will say that the meal tasted good. The problem is that politeness violates a principles of an informative speaker, i.e. the accurate transfer of information. Polite utterances might be misleading, but they are necessary to preserve interpersonal relationships. Something artificial intelligence has not mastered yet, as it usually is programmed in such a way to be efficient and accurate in communication. \\ In the original study the research question was whether the trade-off in conversations between information transfer and being polite could be predicted by a computational model. Therefore Yoon et al. \cite{yoon2016talking} conducted three different experiments, which had three different goals. In the first one they asked participants on their impressions of different utterances (\textit{terrible, bad, okay, good, amazing}) to have a baseline for their model. To do so they presented a rating on a five-hearts rating scale (e.g. 2 out of 5 hearts) and then asked for example "Do you think the speaker thought the cake was \textbf{amazing}?". The second experiment presented scenarios to the participants including an utterance (e.g. \textit{good}) and a goal (e.g. \textit{being nice}) of the speaker. Participants were then asked about the true state of the world, i.e. was the cake indeed good or did the speaker wanted to be polite? The third experiment asked participants about the speaker's goals. If the given utterance was e.g. \textit{"It was bad"} and the true state 1 out of 5 hearts, the speaker's goal was probably to be honest. \\ We took their findings as a baseline and wanted to find out, if the results of the second experiment would be the same in German. To make sure the replication was comparable to the original study we avoided the use of the German words "du" and "Sie" (\textit{you}), as they are a marker for politeness and make sentences seem more polite (using "Sie") or more casual (using "du"). 

\section{Methods}
Our experiment measured what participants inferred about a true state of the world given an utterance (e.g. "Der Kuchen war okay." \textit{The cake was okay.}) and an intention of the speaker in our fictional scenarios. \\
\textbf{Participants} For the experiment we recruited \textbf{(insert number)} participants by sending the link to all students subscribed to the cognitive science mailing list. In our email we asked only German native speakers to participate in the experiment.\\
\textbf{Stimuli and Design} Because we wanted to replicate this experiment in German we translated the 13 scenarios presented in the original study and made sure to avoid the use of the words "du" and "Sie" (\textit{you} in English) as they are marker for politeness in the German language. The scenarios were all constructed in the same manner. One person performed an action (e.g. baked a cake) and asked another person for their opinion about it. The other person responded using different utterances to rate the performance (it was "furchtbar, schlecht, okay, gut oder hervorragend", \textit{terrible, bad, okay, good, or amazing}) and they had different intentions (or goals: being "gemein, ehrlich oder nett", \textit{mean, honest, or nice}). The names of the persons, utterances and intentions were randomized, but it was controlled that every participant rated each of the 15 (five utterances and three intentions) possible combinations.\\
\textbf{Procedure} The participants read the scenarios (e.g. "Laura hat einen Kuchen gebacken und fragt Hannah wie sie ihn findet.", \textit{Laura baked a cake and asks Hannah about it.}). Then the participants were also shown the speaker's utterance and intention, e.g. "Hannah wollte ehrlich sein und sagt: "Der Kuchen ist gut.", \textit{Hannah wanted to be honest and said: "The cake was good."}. The whole experiment can be viewed at \url{https://pragmatics-exp.netlify.com/}. 

\section{Results}

\section{Discussion}

\section{Conclusion}

\bibliography{references}
\bibliographystyle{ieeetr}



\end{document}
